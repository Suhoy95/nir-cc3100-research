\chapter{CC3100SDK}
\label{cha:design}

В этом разделе рассмотрены программных средств и настройки
среды разработки для работы с комплектом CC3100BOOST и
CC31XXEMUBOOST, рассмотренных в прошлой главе. Конечной целью
данной главы будет запуск нескольких примеров, содержащихся
в комплекте c SDK.

\section{Получение SDK}

Чтобы получить программные инструменты для разработки
(SDK -- \textit{Software Development Kit}),
достаточно найти ссылку на скачивание SDK на одной из страниц
продуктов, после чего мы попадем на страницу выбора
требуемого нам установочника для Windows-системы \cite{cc3100sdk}
(Рисунок \ref{cc3100sdk-page}). На момент написания работы,
последней версией SDK была 1.3.0. После выбора SDK понадобиться
заполнить форму от частного лица и/или лица компании и принять
соответствующие условия использования, предоставляемого продукта.
Как все условия будут выполнены вам придет ссылка для получения
актуальной версии SDK.

\myImage{Странциа получения SDK для CC3100}{cc3100sdk-page}{cc3100sdk-page}

\section{Внутреннее устройство SDK}

Выполнив установку SDK в нужную папку, мы получаем
платформонезависимый драйвер \textit{SimpleLink}, документацию
к нему, его портирование для использования с различными отладочными
установками (В частности, к нашей CC3100BOOST+CC31XXEMUBOOST),
а примеры программ и проектов IDE для изучения возможностей
чипа.

Документация, содержащаяся в SDK, при определенном опыте
может частично или полностью заменить руководство пользователя
CC3100\cite{cc3100userguide} и являться более точным и подробным
справочным материалом по использованию тех или иных возможностей
драйвера. К руководству стоит обращаться для получения первоначального
представления о тех или иных подсистемах и протоколах, с которыми
требуется работать.

Структура SDK:
\vspace{1cm}

\begin{verbatim}
./uninstall.exe                   # деинсталятор SDK
./cc3100_sdk_1_2_0_license.pdf    # лицензионное соглашение
./cc3100_sdk_1_2_0_manifest.html  # Описание включенных компонент
                                  # для соблюдения лицензирования
./cc3100-sdk
    | ./readme.txt
    | ./docs               # Документация к SDK
    | ./examples           # основные примеры программ
    | ./netapps            # Примеры использования app-layer
    | ./oslib              # файлы для разработки на freeRTOS
    | ./platform           # примеры портированные в IDE
    | ./simplelink         # код simplelink-драйвера
    | ./simplelink_extlib  # код сторонних библиотек
    | ./third_party/freertos # исходный код базовой ОС
    | ./tools              # собранные бинарные библиотеки
\end{verbatim}

\clearpage

\section{Настройка среды разработки}

Чтобы начать работу с примерами из SDK, достаточно проследовать
руководству по быстрому старту CC3100BOOST\cite{cc3100boostgetstart}.
В ходе которого необходимо убедится, что устройство определилось на
компьютере правильно и воспринимается как серия COM портов,
работающих поверх USB (Рисунок \ref{cc3100sdk-env}).

\mySecondImage{cc3100sdk-env}{cc3100sdk-env}{cc3100sdk-env}{0.6}

Как только устройство начало определяться верно,
можно приступить к скачиванию и установке:
\begin{itemize}
    \item MinGW -- открытого набора компиляторов;
    \item Eclipse -- среда разработки.
\end{itemize}

Далее остается только открыть интересующий нас Exlipse-проект
из папки \textit{cc3100-sdk/platforms}, после чего потребуется
его предварительная настройка, такая как указание
необходимых путей для работы с MinGW, а также
работы Eclipse с внутренним Makefile-ом, который
описывает порядок сборки проекта. При его детальном рассмотрении
можно провести преобразование проекта в более удобный
для разработчика вид.

%%% Local Variables:
%%% mode: latex
%%% TeX-master: "rpz"
%%% End:

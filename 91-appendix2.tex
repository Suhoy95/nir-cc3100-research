\chapter{Исходный код программ}
\label{cha:appendix2}



\section{Тестирование пропускной способности}

\url{https://bitbucket.org/cpractice/cc3100-test}

\begin{verbatim}
#!/usr/bin/env python
import argparse
import socket
import sys
import time


def parse_args():
    parser = argparse.ArgumentParser()
    parser.add_argument('-f', '--filename', type=str, required=True)
    return parser.parse_args()


def start_server():
    server = socket.socket(socket.AF_INET, socket.SOCK_STREAM)
    server.bind(('0.0.0.0', 8080))
    server.listen()
    print('[server] start listening', file=sys.stderr)
    (client, address) = server.accept()  # : :type client: socket.socket
    print('[server][%s:%d] get connectinon.' % address, file=sys.stderr)
    start = time.time()

    data = []
    recived_bytes = 0
    while True:
        part = client.recv(4096)
        if part == b'':
            break
        recived_bytes += len(part)
        data.append(part)

    end = time.time()
    client.close()
    print('[server][%s:%d] close connection. Recievd %f Mb; Time: %f' % (
        address[0], address[1], recived_bytes / (1024 * 1024), end - start),
        file=sys.stderr)

    return (data, recived_bytes / (1024 * 1024), end - start)


def check_data(data, src_filename):
    with open(src_filename, mode='rb') as file:
        for part in data:
            origin_part = file.read(len(part))
            if part != origin_part:
                print('[server][ERROR] Data check is faild.', file=sys.stderr)
                sys.exit(status=-1)
    print('[server] Data check is passed.', file=sys.stderr)


def main():
    args = parse_args()
    (data, size, recv_time) = start_server()
    check_data(data, args.filename)
    print(size, recv_time)


if __name__ == '__main__':
    main()
\end{verbatim}


\begin{verbatim}
#include "header.h"


void setUP_and_check() {
        int32_t retVal = configureSimpleLinkToDefaultState(defaultDevName());
        if(retVal < 0) {
                DEBUG("Failed to configure in default state; Error: %ld", retVal);
                exit(-1);
        }
        DEBUG("Device is configured in default state");

        retVal = sl_Start(0, defaultDevName(), 0);
        if (retVal < 0 || ROLE_STA != retVal) {
                DEBUG("Failed to start the device; Error: %ld", retVal);
                exit(-1);
        }
        DEBUG("Device started as STATION");

        retVal = connectToAP("labi2", SL_SEC_TYPE_WPA_WPA2, "labi2_wpa");
        if(retVal < 0) {
                DEBUG("Failed to establish connection with AP; Error: %ld", retVal);
                exit(-1);
        }
        DEBUG("Connection established with AP and IP is acquired");

        DEBUG("Checking LAN connection - pinging Gateway...!");
        retVal = pingTargetHost(g_GatewayIP);
        if(retVal < 0) {
                DEBUG("Device couldn't connect to LAN; Error: %ld", retVal);
                exit(-1);
        }
        DEBUG("Device successfully connected");
}


int main(int argscount, char *args[])
{
        if(argscount < 3) {
                fprintf(stderr, 'not enough arguments\n');
                exit(-1);
        }

        setUP_and_check();

        uint64_t len = (uint64_t)atoi(args[1]);
        const char * filename = args[2];
        DEBUG("Read file: %s, filesize: %d", filename, len);
        FILE* f = fopen(filename, "rb");
        void* data = malloc(len);
        fread(data, 1, len, f);
        close(f);
        DEBUG("data has been read");

        int32_t sockId = sl_Socket(SL_AF_INET, SL_SOCK_STREAM, SL_IPPROTO_TCP);
        assert(sockId >= 0);

        SlSockAddrIn_t addr = {
                .sin_family = SL_AF_INET,
                .sin_port = sl_Htons(8080),
                .sin_addr.s_addr = sl_Htonl(
                        // 0xC0A80064 // 192.168.0.100
                        0xC0A80067 // 192.168.0.103
                        // 0xC0A82B48 // 192.168.43.72
                        ),
        };
        int16_t status = sl_Connect(sockId, (SlSockAddr_t *) &addr, sizeof(SlSockAddrIn_t));
        assert(status >= 0);
        DEBUG("connected");

        for(int32_t i = 0; i < len / 1024; i++) {
                status = sl_Send(sockId, data + i * 1024, 1024, 0);
                assert(status > 0);
        }
        sl_Close(sockId);

        DEBUG("data sended");
        return 0;
}
\end{verbatim}

%%% Local Variables:
%%% mode: latex
%%% TeX-master: "rpz"
%%% End:

\chapter{Тестирование пропускной способности CC3100}
\label{cha:impl}


\begin{verbatim}
int main(int argscount, char *args[])
{
    if(argscount < 3) {
        fprintf(stderr, 'not enough arguments\n');
        exit(-1);
    }

    setUP_and_check();

    uint64_t len = (uint64_t)atoi(args[1]);
    const char * filename = args[2];
    DEBUG("Read file: %s, filesize: %d", filename, len);
    FILE* f = fopen(filename, "rb");
    void* data = malloc(len);
    fread(data, 1, len, f);
    close(f);
    DEBUG("data has been read");

    int32_t sockId = sl_Socket(SL_AF_INET, SL_SOCK_STREAM, SL_IPPROTO_TCP);
    assert(sockId >= 0);

    SlSockAddrIn_t addr = {
            .sin_family = SL_AF_INET,
            .sin_port = sl_Htons(8080),
            .sin_addr.s_addr = sl_Htonl(
                    // 0xC0A80064 // 192.168.0.100
                    0xC0A80067 // 192.168.0.103
                    // 0xC0A82B48 // 192.168.43.72
                    ),
    };
    int16_t status = sl_Connect(sockId, (SlSockAddr_t *) &addr, sizeof(SlSockAddrIn_t));
    assert(status >= 0);
    DEBUG("connected");

    for(int32_t i = 0; i < len / 1024; i++) {
            status = sl_Send(sockId, data + i * 1024, 1024, 0);
            assert(status > 0);
    }
    sl_Close(sockId);

    DEBUG("data sended");
    return 0;
}
\end{verbatim}

\begin{verbatim}
def start_server():
    server = socket.socket(socket.AF_INET, socket.SOCK_STREAM)
    server.bind(('0.0.0.0', 8080))
    server.listen()
    print('[server] start listening', file=sys.stderr)
    (client, address) = server.accept()  # : :type client: socket.socket
    print('[server][%s:%d] get connectinon.' % address, file=sys.stderr)
    start = time.time()

    data = []
    recived_bytes = 0
    while True:
        part = client.recv(4096)
        if part == b'':
            break
        recived_bytes += len(part)
        data.append(part)

    end = time.time()
    client.close()
    print('[server][%s:%d] close connection. Recievd %f Mb; Time: %f' % (
        address[0], address[1], recived_bytes / (1024 * 1024), end - start),
        file=sys.stderr)

    return (data, recived_bytes / (1024 * 1024), end - start)
\end{verbatim}

\myImage{bandwidth-test}{bandwidth-test}{bandwidth-test}

%%% Local Variables:
%%% mode: latex
%%% TeX-master: "rpz"
%%% End:

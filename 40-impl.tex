\chapter{Тестирование пропускной способности CC3100}
\label{cha:impl}

Последним этапом этой работы, является тестирование
заявленной пропускной способности, которая судя по
разделу \ref{cha:analysis} должна быть около 13 Мбит в секунду.

Для этого возьмем пример из SDK, который демонстрирует
работу CC3100 в качестве пользовательской станции и производит
проверку наличия соединения с сетью Интернет с помощью ICMP\cite{icmp}.

Подкорректировав предварительную настройку и разместив ее в
функцию \textmd{setUP\_and\_check()}, нам остается только дописать
работу с \textit{berkley-like} сокетом для соединения с
тестирующим скорость TCP-сервером и передачи файла, название
которого задается через командную строку.

Сам TCP-сервер написан на интерпретируемом языке \textit{Python}
версии 3.5. Его основная задача заключается в циклическом
ожидании подключения устройства, получения тестовых данных
и их сверку с эталонным файлом, чтобы убедится в том, что
данные передались правильно и целостно. Подробнее
исходный код приложения можно изучить в приложении \ref{cha:appendix2}.

\clearpage
Основная логика приложения, работающая с CC3100:

\begin{verbatim}
int main(int argscount, char *args[])
{
    setUP_and_check();

    // считывание тестового файла
    uint64_t len = (uint64_t)atoi(args[1]);
    const char * filename = args[2];
    FILE* f = fopen(filename, "rb");
    void* data = malloc(len);
    fread(data, 1, len, f);
    close(f);

    // создание соединения
    int32_t sockId = sl_Socket(SL_AF_INET, SL_SOCK_STREAM,
                                            SL_IPPROTO_TCP);

    SlSockAddrIn_t addr = {
            .sin_family = SL_AF_INET,
            .sin_port = sl_Htons(8080),
            .sin_addr.s_addr = sl_Htonl(
                    0xC0A80067 // 192.168.0.103
                    ),
    };
    int16_t status = sl_Connect(sockId, (SlSockAddr_t *) &addr,
                                        sizeof(SlSockAddrIn_t));
    // передача данных на TCP сервер
    for(int32_t i = 0; i < len / 1024; i++) {
            status = sl_Send(sockId, data + i * 1024, 1024, 0);
    }
    sl_Close(sockId);
}
\end{verbatim}


\myImage{Результаты тестирования пропускной способности
в зависимости от дальности передачи данных}{bandwidth-test}{bandwidth-test}

Модернизировав полученный пример выводом дополнительных сведений
о соединении, а также включив код измерения пропускной
способности в среднем, был проведен ряд измерений для
наблюдения пропускной способности в зависимости от расстояния.
Полученные результаты можно увидеть на графике \ref{bandwidth-test},
на котором можно наблюдать, что пропускная способность держится
в диапазоне от 10 да 12 Мбит в секунду, что близко к заявленной.


%%% Local Variables:
%%% mode: latex
%%% TeX-master: "rpz"
%%% End:

\Introduction

Целью данной работы является изучение возможностей и
структуры wifi-чипа \textit{СС3100}\cite{cc3100}, разработанного и
поставляемого компанией \textit{Texas Instruments}, для использования
в качестве передатчика информации в рамках IoT-решений
(\textit{IoT -- Internet of Things -- Интернет вещей}).
Данное исследование является предварительным сбором сведений
для возможной интеграции данного чипа с миниатюрными диктофонами
ООО <<Вторая лаборатория>>\cite{labi2dicts}, с целью
осуществления удаленной настройки диктофона и получения
уже записанных аудиоданных на диктофон. Поэтому одними
из главных исследуемых характеристик будет пропускная способность
и низкое энергопотребление, как при работе чипа, так и при
выключенном (спящем) режиме.

Чип CC3100 создан для предоставления готового стека
Интернет протоколов в устройства с отдельным модулем управления
\textit{MCU} (\textit{Microcontroller Unit}), поэтому для
проведения требуемых измерений помимо самого чипа
требуется какая-либо установка, позволяющая предоставить
требуемое питание и внешнее программное управление.
Для этих целей компания Texas Instruments поставляет
установки CC3100BOOST\cite{cc3100boost} и CC31XXEMUBOOST\cite{cc31xxemuboost},
а также прилагает к ним программные инструменты для разработки
\textit{SDK}(\textit{Software Development Kit}) для запуска
отладочных управляющих программ на стационарном компьютере,
которые позволяют проверить заявленные чипом CC3100 характеристики.

\clearpage
Таким образом, для достижения поставленной цели необходимо решить следующие задачи:

\begin{itemize}
\item Изучить документацию и рекомендации по разработке, прилагающееся к CC3100;
\item Ознакомиться с устройством и документацией схемы CC3100BOOST;
\item Ознакомиться с устройством и документацией схемы CC31XXEMUBOOST;
\item Собрать на основе схем CC3100BOOST и CC31XXEMUBOOST отладочную установку;
\item Запустить на полученной установке примеры программ из SDK, прилагающегося к документации;
\item Провести тестирование пропускной способности CC3100, используя рассмотренные примеры
программ и руководства к разработке.
\end{itemize}
